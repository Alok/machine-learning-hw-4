\usepackage{mathrsfs,listings,hyperref,backref,amsmath,amsfonts,textcomp,amssymb,geometry,graphicx,enumerate,algorithm,algorithmicx,pdfsync}
\usepackage[noend]{algpseudocode}
\usepackage[T1]{fontenc}
\usepackage[latin9]{inputenc}
\usepackage[parfill]{parskip}
\graphicspath{{./figs/}}
\pagenumbering{roman}
\synctex=1

\def\Name{Alok Singh}
\def\SID{24456212}
\def\Homework{4}
\def\Session{Spring 2016}
\def\Class{CS 189}

\title{\Class --- \Session --- HW \Homework\ Solutions} \author{\Name, SID
\SID}
\date{}
\textheight=9in
\textwidth=6.5in
\topmargin=-.75in
\leftmargin=0.1in
\rightmargin=0.1in
\oddsidemargin=0.25in
\evensidemargin=0.25in

% Custom commands go here
\let\bf\textbf
\newcommand{\ep}{\varepsilon}
\newcommand{\f}{\forall}

\begin{document}
\maketitle

\section{1}
\label{sec:1}

\section{2}
\label{sec:2}

\section{3}
\label{sec:3}

\section{4}
\label{sec:4}

\begin{enumerate}[(a)]
    \item If we note that $\tanh = \frac{\sinh}{\cosh}$, then using the definitions of $\sinh$ and $\cosh$ gives us the desired statement, as the factor of $\frac{1}{2}$ in $\sinh$ and $\cosh$ disappears.

    \item $g'$ is just $\frac{\tanh\'}{2}$ as factor of a half vanishes after differentiating. Using the analogue of the $\tan^2+\sec^2=1$ for hyperbolic functions, we get that the derivative is $\frac{1-\tanh^2}{2}$

    \item 
\end{enumerate}

\section{5}
\label{sec:5}



\end{document}
